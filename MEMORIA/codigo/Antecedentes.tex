\chapter{Estado del arte}
\label{cap:Antecedentes}

El objetivo de este capítulo consistirá en la definición y explicación detallada del argumento del presente TFG. A continuación, se presenta el estado del arte del método de desarrollo DGS, donde se detallará su definición, beneficios, problemáticas. Además, se indicará la importancia que conlleva la gestión en un proyecto de estas características. Por otra parte, se harán referencia a las habilidades que son necesarias para llevar a cabo el trabajo de jefe de proyecto en un entorno DGS. A continuación, se tratará la situación actual de la gamificación para la enseñanza, junto con los JS y el uso de la IA para la personalización de juegos. Para finalizar, se listarán una serie de ejemplos relacionados con el objetivo de este TFG.

\section{Desarrollo Global del Software}
\label{sec:DGS}

Debido a la globalización, en el campo de la ingeniería de software aparece un nuevo modelo de desarrollo denominado como Desarrollo Global de Software. A diferencia del modelo tradicional, donde el equipo de trabajo estaba centralizado en un solo edificio o en varios edificios, pero siempre en el mismo país, el DGS consiste en el desarrollo de un producto o servicio software, en donde diferentes equipos de desarrollo, pertenecientes a organizaciones diferentes y ubicadas en países dispares, colaboran en el mismo proyecto software y son coordinados y gestionados en tiempo real basándose en los desafíos del DGS, las 3Cs: comunicación, coordinación y control \cite{piattini2014desarrollo}.

A lo largo de la última década, el DGS se ha afianzado como una de las vertientes más relevantes en la investigación y práctica dentro del campo de la ingeniería del software. Las primeras prácticas de este nuevo tipo de proceso de desarrollo de software surgen hace más de 30 años, con los primeros usos de \emph{outsourcing} \cite{boehm2006view}. Sin embargo, no sería hasta 2006 cuando se extendiera su uso con la celebración de la primera conferencia internacional sobre DGS, el \emph{ICGSE} (\emph{IEEE International Conference on Global Software Engineering}) \cite{piattini2014desarrollo, vizcaino2015vision}.

\subsection{Beneficios del Desarrollo Global del Software}
\label{sec:Beneficios}

Cuando hablamos de DGS es notable resaltar la cantidad de beneficios que pueden alcanzar las empresas con su correcto uso. Sin embargo, no todos sus beneficios son conocidos en el sector, ya que podemos encontrar beneficios que se puedan obtener de forma directa o indirecta, y son muchas las investigaciones, como \cite{aagerfalk2008benefits, conchuir2009global, conchuir2006exploring, vizcaino2015vision}, que están estudiado tanto aquellos beneficios notables como aquellos no vistos a simple vista con el uso de DGS como proceso de desarrollo de software. Por lo tanto, a continuación se formulará un listado de todos los beneficios que podemos obtener con esta nueva tendencia:

\begin{itemize}
	\item \textbf{Reducción del coste de producción.} Uno de los beneficios más importantes y una de las razones por las cuales las organizaciones y compañías están optando en afrontar un modelo de desarrollo de software global, consiste en la reducción de los costes del proyecto en el proceso de producción. Este beneficio se debe en gran medida a la globalización, ya que ha hecho posible que actividades en el proceso de desarrollo puedan realizarlas empleados que se encuentran en países que cuentan con salarios más reducidos \cite{aagerfalk2008benefits}. Un ejemplo de esta situación sería la India, en donde el salario base anual de un desarrollador de software es de \emph{US\$15,000}, mientras que en Irlanda un mismo trabajador puede ganar cuatro veces esa cantidad, y a su vez esa cantidad consistiría en la mitad del salario de un desarrollador en Estados Unidos \cite{conchuir2009global, conchuir2006exploring}. Alcanzar esta situación también ha sido posible gracias al despliegue de enlaces de comunicación de alta velocidad, los cuales ayudan a la transferencia de información y conocimientos entre los empleados separados geográficamente \cite{aagerfalk2008benefits}. Este beneficio conlleva que las compañías tengan que tener en cuenta la gestión del coste en cuanto a los viajes de empleados entre grupos de trabajo, ya que en este modelo de desarrollo los empleados no se conocen y es necesario que exista un poco de comunicación \emph{face-to-face} para consolidar la relación entre empleados, creando así un ambiente de confianza \cite{conchuir2009global, conchuir2006exploring}.
	
	\item \textbf{Aprovechamiento de la zona horaria.} En un entorno global, como es un proyecto que utiliza DGS como proceso de desarrollo, hace posible que las organizaciones puedan aprovecharse de la diferencia en las zonas horarias de sus empleados, con el fin de incrementar las horas de trabajo del día y reducir así el tiempo de desarrollo del servicio o producto software \cite{conchuir2006exploring, conchuir2009global}. De esta manera se lograrán jornadas de trabajo más extensas en el proyecto, que en uno con un modelo tradicional y por lo tanto una mayor productividad para finalizar dicho proyecto en un menor tiempo \cite{vizcaino2015vision}. Esta situación es conocida como desarrollo \emph{follow-the-sun}\footnote{"La estrategia follow-the-sun se caracteriza en que cuando un equipo de trabajo finaliza su jornada laboral, la jornada de otro equipo comienza en otra parte del mundo, de esta manera se consigue un desarrollo del proyecto las 24 horas del día" \cite{piattini2014desarrollo}} y es considerada un potente beneficio en el DGS. Sin embargo, lograr este escenario en la realidad es complicado, además de poder producirse retrasos en las respuestas en la comunicación de los empleados, debido a horarios de comida o días festivos, lo que puede ocasionar retrasos en el proyecto. Por lo tanto, es importante que exista cierto solapamiento en las jornadas laborales de los empleados, con el fin de que exista cierta comunicación sincrona \cite{conchuir2006exploring, conchuir2009global}.
	
	\item \textbf{Modularización del proceso de desarrollo.} Una manera de afrontar un proyecto DGS es separando las tareas del proyecto en módulos independientes bien definidos, para que de esta manera cada equipo de desarrollo posea ciertos módulos de trabajo. Permitiendo así que las decisiones referentes a cada módulo se tomen de forma aislada entre los miembros del equipo de trabajo, además de reducir costes en la coordinación. De esta manera, podemos diferenciar dos tipos de estrategias: \emph{basada en módulos}\footnote{"La estrategia basada en módulos consiste en dividir el proyecto en diferentes módulos, los cuales se pueden considerar un artefacto completo del proyecto, y repartirlos entre los sites" \cite{piattini2014desarrollo}} y \emph{basada en fases}\footnote{"La estrategia basada en fases consiste en asignar a cada equipo de trabajo una fase del proceso de desarrollo software" \cite{piattini2014desarrollo}} \cite{conchuir2006exploring, conchuir2009global, aagerfalk2008benefits}.
	
	\item \textbf{Acceso a plantillas de trabajadores altamente cualificados.} El DGS hace posible acceder fácilmente a grupos de trabajadores altamente cualificados repartidos por todo el mundo. De esta manera, las empresas se pueden beneficiar de los conocimientos, diversidad de experiencias, destrezas y habilidades de trabajadores repartidos por todo el mundo, para que lleven a cabo el desarrollo de diferentes actividades software del proyecto DGS \cite{vizcaino2015vision, conchuir2006exploring, conchuir2009global, aagerfalk2008benefits}.
	
	\item \textbf{Proximity to market and costumer.} Gracias al DGS, las compañías pueden establecer fácilmente filiales en aquellos países donde se localizan los clientes, y conseguir así un acercamiento y un conocimiento más profundo del mercado local. Además, con esta practica las compañías consiguen expandirse hacia nuevos mercados, sin la necesidad de trasladar a sus equipos de desarrolladores \cite{vizcaino2015vision, conchuir2006exploring, conchuir2009global, aagerfalk2008benefits}.
\end{itemize}

\subsection{Desafíos del Desarrollo Global del Software}
\label{sec:Desafios}

\subsection{Rol del jefe de proyecto}
\label{sec:ImportanciaJP}

\section{Habilidades necesarias en Desarrollo Global del Software}
\label{sec:HabilidadesDGS}

\subsection{Habilidades en el equipo de trabajo de Desarrollo Global del Software}
\label{sec:HabilidadesT}

\subsection{Habilidades en jefes de proyecto de Desarrollo Global del Software}
\label{sec:HabilidadesJP}

\section{Gamificación}
\label{sec:Gamificacion}

\subsection{Juegos Serios}
\label{sec:JuegosSerios}

\section{Trabajos relacionados con el tema}
\label{sec:TrabajosRelacionados}

\subsection{Juegos Serios para Desarrollo Global del Software}
\label{sec:JuegosSeriosDGS}

\subsection{Juegos Serios para Jefes de Proyecto}
\label{sec:JuegosSeriosJP}

\subsection{Juegos Serios para Jefes de Proyecto en Desarrollo Global del Software}
\label{sec:JuegosSeriosJPDGS}