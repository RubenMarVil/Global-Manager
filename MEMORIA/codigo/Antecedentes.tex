\chapter{Estado del arte}
\label{cap:Antecedentes}

El objetivo de este capítulo consistirá en la definición y explicación detallada del argumento del presente TFG. A continuación, se presenta el estado del arte del método de desarrollo DGS, donde se detallará su definición, beneficios, problemáticas. Además, se indicará la importancia que conlleva la gestión en un proyecto de estas características. Por otra parte, se harán referencia a las habilidades que son necesarias para llevar a cabo el trabajo de jefe de proyecto en un entorno DGS. A continuación, se tratará la situación actual de la gamificación para la enseñanza, junto con los JS y el uso de la IA para la personalización de juegos. Para finalizar, se listarán una serie de ejemplos relacionados con el objetivo de este TFG.

\section{Desarrollo Global del Software}
\label{sec:DGS}

Debido a la globalización, en el campo de la ingeniería de software aparece un nuevo modelo de desarrollo denominado como Desarrollo Global de Software. A diferencia del modelo tradicional, donde el equipo de trabajo estaba centralizado en un solo edificio o en varios edificios, pero siempre en el mismo país, el DGS consiste en el desarrollo de un producto o servicio software, en donde diferentes equipos de desarrollo, pertenecientes a organizaciones diferentes y ubicadas en países dispares, colaboran en el mismo proyecto software y son coordinados y gestionados en tiempo real basándose en los desafíos del DGS, las 3Cs: comunicación, coordinación y control \cite{piattini2014desarrollo}.

A lo largo de la última década, el DGS se ha afianzado como una de las vertientes más relevantes en la investigación y práctica dentro del campo de la ingeniería del software. Las primeras prácticas de este nuevo tipo de proceso de desarrollo de software surgen hace más de 30 años, con los primeros usos de \emph{outsourcing} \cite{boehm2006view}. Sin embargo, no sería hasta 2006 cuando se extendiera su uso con la celebración de la primera conferencia internacional sobre DGS, el \emph{ICGSE} (\emph{IEEE International Conference on Global Software Engineering}) \cite{piattini2014desarrollo, vizcaino2015vision}.

\subsection{Beneficios del Desarrollo Global del Software}
\label{sec:Beneficios}

Cuando hablamos de DGS es notable resaltar la cantidad de beneficios que pueden alcanzar las empresas con su correcto uso. Sin embargo, no todos sus beneficios son conocidos en el sector, ya que podemos encontrar beneficios que se puedan obtener de forma directa o indirecta, y son muchas las investigaciones, como \cite{aagerfalk2008benefits, conchuir2009global, conchuir2006exploring, vizcaino2015vision}, que están estudiado tanto aquellos beneficios notables como aquellos no vistos a simple vista con el uso de DGS como proceso de desarrollo de software. Por lo tanto, a continuación se formulará un listado de todos los beneficios que podemos obtener con esta nueva tendencia:

\begin{itemize}
	\item \textbf{Reducción del coste de producción.} Uno de los beneficios más importantes y una de las razones por las cuales las organizaciones y compañías están optando en afrontar un modelo de desarrollo de software global, consiste en la reducción de los costes del proyecto en el proceso de producción. Este beneficio se debe en gran medida a la globalización, ya que ha hecho posible que actividades en el proceso de desarrollo puedan realizarlas empleados que se encuentran en países que cuentan con salarios más reducidos \cite{aagerfalk2008benefits}. Un ejemplo de esta situación sería la India, en donde el salario base anual de un desarrollador de software es de \emph{US\$15,000}, mientras que en Irlanda un mismo trabajador puede ganar cuatro veces esa cantidad, y a su vez esa cantidad consistiría en la mitad del salario de un desarrollador en Estados Unidos \cite{conchuir2009global, conchuir2006exploring}. Alcanzar esta situación también ha sido posible gracias al despliegue de enlaces de comunicación de alta velocidad, los cuales ayudan a la transferencia de información y conocimientos entre los empleados separados geográficamente \cite{aagerfalk2008benefits}. Este beneficio conlleva que las compañías tengan que tener en cuenta la gestión del coste en cuanto a los viajes de empleados entre grupos de trabajo, ya que en este modelo de desarrollo los empleados no se conocen y es necesario que exista un poco de comunicación \emph{face-to-face} para consolidar la relación entre empleados, creando así un ambiente de confianza \cite{conchuir2009global, conchuir2006exploring}.
	
	\item \textbf{Aprovechamiento de la zona horaria.} En un entorno global, como es un proyecto que utiliza DGS como proceso de desarrollo, hace posible que las organizaciones puedan aprovecharse de la diferencia en las zonas horarias de sus empleados, con el fin de incrementar las horas de trabajo del día y reducir así el tiempo de desarrollo del servicio o producto software \cite{conchuir2006exploring, conchuir2009global}. De esta manera se lograrán jornadas de trabajo más extensas en el proyecto, que en uno con un modelo tradicional y por lo tanto una mayor productividad para finalizar dicho proyecto en un menor tiempo \cite{vizcaino2015vision}. Esta situación es conocida como desarrollo \emph{follow-the-sun}\footnote{"La estrategia follow-the-sun se caracteriza en que cuando un equipo de trabajo finaliza su jornada laboral, la jornada de otro equipo comienza en otra parte del mundo, de esta manera se consigue un desarrollo del proyecto las 24 horas del día" \cite{piattini2014desarrollo}} y es considerada un potente beneficio en el DGS. Sin embargo, lograr este escenario en la realidad es complicado, además de poder producirse retrasos en las respuestas en la comunicación de los empleados, debido a horarios de comida o días festivos, lo que puede ocasionar retrasos en el proyecto. Por lo tanto, es importante que exista cierto solapamiento en las jornadas laborales de los empleados, con el fin de que exista cierta comunicación sincrona \cite{conchuir2006exploring, conchuir2009global}.
	
	\item \textbf{Modularización del proceso de desarrollo.} Una manera de afrontar un proyecto DGS es separando las tareas del proyecto en módulos independientes bien definidos, para que de esta manera cada equipo de desarrollo posea ciertos módulos de trabajo. Permitiendo así que las decisiones referentes a cada módulo se tomen de forma aislada entre los miembros del equipo de trabajo, además de reducir costes en la coordinación. De esta manera, podemos diferenciar dos tipos de estrategias: \emph{basada en módulos}\footnote{"La estrategia basada en módulos consiste en dividir el proyecto en diferentes módulos, los cuales se pueden considerar un artefacto completo del proyecto, y repartirlos entre los sites" \cite{piattini2014desarrollo}} y \emph{basada en fases}\footnote{"La estrategia basada en fases consiste en asignar a cada equipo de trabajo una fase del proceso de desarrollo software" \cite{piattini2014desarrollo}} \cite{conchuir2006exploring, conchuir2009global, aagerfalk2008benefits}.
	
	\item \textbf{Acceso a plantillas de trabajadores altamente cualificados.} El DGS hace posible acceder fácilmente a grupos de trabajadores altamente cualificados repartidos por todo el mundo. De esta manera, las empresas se pueden beneficiar de los conocimientos, diversidad de experiencias, destrezas y habilidades de trabajadores repartidos por todo el mundo, para que lleven a cabo el desarrollo de diferentes actividades software del proyecto DGS \cite{vizcaino2015vision, conchuir2006exploring, conchuir2009global, aagerfalk2008benefits}.
	
	\item \textbf{Proximity to market and costumer.} Gracias al DGS, las compañías pueden establecer fácilmente filiales en aquellos países donde se localizan los clientes, y conseguir así un acercamiento y un conocimiento más profundo del mercado local. Además, con esta practica las compañías consiguen expandirse hacia nuevos mercados, sin la necesidad de trasladar a sus equipos de desarrolladores \cite{vizcaino2015vision, conchuir2006exploring, conchuir2009global, aagerfalk2008benefits}.
\end{itemize}

\subsection{Desafíos del Desarrollo Global del Software}
\label{sec:Desafios}

La distancia existente entre los equipos de desarrolladores en un proyecto DGS hace posible el acercamiento hacia grandes beneficios por parte de las organizaciones, sin embargo conseguirlos no es tarea sencilla, ya que dicha distancia conlleva también la introducción de un conjunto de desafíos, a los cuales, las organizaciones deben hacer frente para alcanzar los beneficios anteriormente citados \cite{conchuir2006exploring}. Los desafíos que puedan aparecer en este tipo de proyectos se pueden agrupar en relación con los tres grandes procesos en el desarrollo de software: comunicación, coordinación y control. Estos grupos de desafíos también se les conoce como las 3 Cs \cite{vizcaino2015vision, piattini2014desarrollo}, y hacen referencia a las siguientes situaciones:

\begin{itemize}
	\item \textbf{Desafíos de comunicación.} Hace referencia a aquellas situaciones en donde se lleva a cabo una comunicación entre trabajadores, es decir, un intercambio de información y conocimientos con el fin de que no se produzcan malentendidos y el proyecto pueda avanzar.
	\item \textbf{Desafíos de coordinación.} Hace referencia al mantenimiento de los trabajadores en la realización de las diferentes tareas de un proyecto, con el fin de alcanzar objetivos e intereses comunes y la evolución del proyecto progrese adecuadamente.
	\item \textbf{Desafíos de control.} Hace referencia a la administración y gestión del proyecto en general, teniendo en cuenta día a día diferentes aspectos del proyecto como pueden ser los calendarios de entregas, presupuestos, calidad, estándares, etc. 
\end{itemize}

Por otro lado, los proyectos DGS se caracterizan por la existencia de diferentes nacionalidades, organizaciones e inclusos culturas que pueden agravar esta situación. Esto hace que puedan aparecer diferentes tipos de distancias entre los miembros del proyecto, provocando una acentuación en los desafíos anteriormente citados \cite{vizcaino2015vision}. De esta manera, aparece otra clasificación de los desafíos de un proyecto DGS en función de lo que se conoce en la literatura como las tres distancias \cite{vizcaino2015vision, conchuir2006exploring, conchuir2009global}:

\begin{itemize}
	\item \textbf{Distancia geográfica.} Se puede definir como la medida de esfuerzo necesario en un individuo para visitar un punto alejado de su ubicación. Un ejemplo sería el de dos ubicaciones separadas geográficamente por una gran distancia pero con un enlace aéreo directo frente a dos ubicaciones cercanas geográficamente pero con poca infraestructura de transporte; de esta manera el primer caso poseerá poca distancia geográfica, mientras que en el segundo será elevada \cite{vizcaino2015vision}. Además, la distancia geográfica dificulta la posibilidad de existir una comunicación informal o cara a cara entre los miembros que ayude a afianzar las relaciones entre ellos, el trabajo en equipo, consolidación de la confianza y la comunicación fluida de información importante del proyecto \cite{conchuir2006exploring}.
	
	\item \textbf{Distancia temporal.} Ligada a la anterior, se puede definir como la medida de la diferencia en el tiempo existente en la comunicación entre dos individuos \cite{vizcaino2015vision}. Como se dijo en la sección anterior, la distancia temporal en un proyecto DGS puede conllevar ciertos desafíos como es la estrategia follow-the-sun, con el objetivo de reducir tiempo y costes, sin embargo, surgen también ciertas problemáticas como es el hecho de que aparezcan retrasos en las respuestas en los intercambios de información entre los trabajadores, debido a la reducción del solapamiento de horas en las jornadas laborales de los equipos de desarrolladores. Esto implica que se tengan que usar herramientas de comunicación asíncronas, las cuales pueden afectar negativamente al manejo de ambigüedades y al aumento de los malentendidos, produciéndose así retrasos en el proyecto; frente a una comunicación síncrona mucho más directa y segura \cite{conchuir2006exploring}.
	
	\item \textbf{Distancia socio-cultural.} Se puede definir como la medida en que un individuo conoce y comprende las costumbres, cultura e idioma de otro individuo, con el objetivo de llevar a cabo una correcta comunicación. Esta situación es debida a que como en un proyecto coexisten diferentes nacionalidades, es frecuente que existan diferentes culturas entre sus miembros \cite{vizcaino2015vision}. Dicha distancia puede conllevar diferentes interpretaciones en una comunicación o situación, lo que puede obstaculizar la comunicación y coordinación del proyecto \cite{conchuir2006exploring}. Además, puede provocar que aparezcan conflictos y malentendidos entre los miembros del proyecto, retrasando así la evolución del mismo.
\end{itemize}

Por lo tanto, es importante tener en cuenta los posibles desafíos que puedan surgir en un proyecto DGS, es por ello que son numerosas las investigaciones sobre estas problemáticas, como es el caso de \cite{niazi2016challenges}, en donde se lleva a cabo una Revisión Sistemática de la Literatura (RSL) de 101 estudios sobre las dificultades más importantes en un proyecto de estas características, junto con un cuestionario a 41 organizaciones sobre sus prácticas en el mundo real. A continuación, en la Tabla \ref{tab:DificultadesDGS}, se mostrará un resumen con las dificultades más importantes y comunes en proyectos DGS.

% Table generated by Excel2LaTeX from sheet 'Hoja1'
\begin{table}[htbp]
  \centering
  \resizebox{\textwidth}{!}{
    \begin{tabular}{lccc|}
    \rowcolor[rgb]{ .851,  .851,  .851} \multicolumn{1}{c}{\textbf{Desafío DGS}} & \multicolumn{1}{p{12em}}{\textbf{Frequencia de aparición \newline{}en 101 estudios (\%)}} & \multicolumn{1}{p{12.93em}}{\textbf{Frecuencia de acuerdo \newline{}con 41 organizaciones (\%)}} & \textbf{Media (\%)} \\
        \rowcolor[rgb]{ .949,  .949,  .949} \textit{Falta de entendimiento cultural} & \cellcolor[rgb]{ 1,  1,  1}62 & \cellcolor[rgb]{ 1,  1,  1}70 & \cellcolor[rgb]{ 1,  1,  1}\textbf{66} \\
        \rowcolor[rgb]{ .949,  .949,  .949} \textit{Ausencia de comunicación} & \cellcolor[rgb]{ 1,  1,  1}54 & \cellcolor[rgb]{ 1,  1,  1}76 & \cellcolor[rgb]{ 1,  1,  1}\textbf{65} \\
        \rowcolor[rgb]{ .949,  .949,  .949} \textit{Falta de la gestión del conocimiento} & \cellcolor[rgb]{ 1,  1,  1}38 & \cellcolor[rgb]{ 1,  1,  1}78 & \cellcolor[rgb]{ 1,  1,  1}\textbf{58} \\
        \rowcolor[rgb]{ .949,  .949,  .949} \textit{Falta de gestión del tiempo} & \cellcolor[rgb]{ 1,  1,  1}42 & \cellcolor[rgb]{ 1,  1,  1}71 & \cellcolor[rgb]{ 1,  1,  1}\textbf{56.5} \\
        \rowcolor[rgb]{ .949,  .949,  .949} \textit{Ausencia de coordinación} & \cellcolor[rgb]{ 1,  1,  1}35 & \cellcolor[rgb]{ 1,  1,  1}69 & \cellcolor[rgb]{ 1,  1,  1}\textbf{52} \\
        \rowcolor[rgb]{ .949,  .949,  .949} \textit{Ausencia de control} & \cellcolor[rgb]{ 1,  1,  1}27 & \cellcolor[rgb]{ 1,  1,  1}75 & \cellcolor[rgb]{ 1,  1,  1}\textbf{51} \\
        \rowcolor[rgb]{ .949,  .949,  .949} \textit{Actividades de ingeniería de requisitos} & \cellcolor[rgb]{ 1,  1,  1}28 & \cellcolor[rgb]{ 1,  1,  1}71 & \cellcolor[rgb]{ 1,  1,  1}\textbf{49.5} \\
        \rowcolor[rgb]{ .949,  .949,  .949} \textit{Asignación de tareas} & \cellcolor[rgb]{ 1,  1,  1}18 & \cellcolor[rgb]{ 1,  1,  1}80 & \cellcolor[rgb]{ 1,  1,  1}\textbf{49} \\
        \rowcolor[rgb]{ .949,  .949,  .949} \textit{Ausencia de la verdad} & \cellcolor[rgb]{ 1,  1,  1}34 & \cellcolor[rgb]{ 1,  1,  1}59 & \cellcolor[rgb]{ 1,  1,  1}\textbf{46.5} \\
        \rowcolor[rgb]{ .949,  .949,  .949} \textit{Actividades de gestión del cambio} & \cellcolor[rgb]{ 1,  1,  1}22 & \cellcolor[rgb]{ 1,  1,  1}68 & \cellcolor[rgb]{ 1,  1,  1}\textbf{45} \\
        \rowcolor[rgb]{ .949,  .949,  .949} \textit{Falta en la concienciación de equipo} & \cellcolor[rgb]{ 1,  1,  1}23 & \cellcolor[rgb]{ 1,  1,  1}66 & \cellcolor[rgb]{ 1,  1,  1}\textbf{44.5} \\
        \rowcolor[rgb]{ .949,  .949,  .949} \textit{Gestión de conflictos} & \cellcolor[rgb]{ 1,  1,  1}17 & \cellcolor[rgb]{ 1,  1,  1}71 & \cellcolor[rgb]{ 1,  1,  1}\textbf{44} \\
        \rowcolor[rgb]{ .949,  .949,  .949} \textit{Estimación del coste y del esfuerzo} & \cellcolor[rgb]{ 1,  1,  1}15 & \cellcolor[rgb]{ 1,  1,  1}73 & \cellcolor[rgb]{ 1,  1,  1}\textbf{44} \\
        \rowcolor[rgb]{ .949,  .949,  .949} \textit{Actividades de integración} & \cellcolor[rgb]{ 1,  1,  1}14 & \cellcolor[rgb]{ 1,  1,  1}73 & \cellcolor[rgb]{ 1,  1,  1}\textbf{43.5} \\
        \rowcolor[rgb]{ .949,  .949,  .949} \textit{Gestión del riesgo} & \cellcolor[rgb]{ 1,  1,  1}15 & \cellcolor[rgb]{ 1,  1,  1}71 & \cellcolor[rgb]{ 1,  1,  1}\textbf{43} \\
        \rowcolor[rgb]{ .949,  .949,  .949} \textit{Distancia geográfica} & \cellcolor[rgb]{ 1,  1,  1}28 & \cellcolor[rgb]{ 1,  1,  1}56 & \cellcolor[rgb]{ 1,  1,  1}\textbf{42} \\
        \rowcolor[rgb]{ .949,  .949,  .949} \textit{Ausencia de un proceso uniforme entre los diferentes sites} & \cellcolor[rgb]{ 1,  1,  1}19 & \cellcolor[rgb]{ 1,  1,  1}63 & \cellcolor[rgb]{ 1,  1,  1}\textbf{41} \\
        \rowcolor[rgb]{ .949,  .949,  .949} \textit{Falta de una infraestructura informatica adecuada} & \cellcolor[rgb]{ 1,  1,  1}11 & \cellcolor[rgb]{ 1,  1,  1}68 & \cellcolor[rgb]{ 1,  1,  1}\textbf{39.5} \\
        \rowcolor[rgb]{ .949,  .949,  .949} \textit{Protección de la propiedad intelectual} & \cellcolor[rgb]{ 1,  1,  1}9 & \cellcolor[rgb]{ 1,  1,  1}64 & \cellcolor[rgb]{ 1,  1,  1}\textbf{36.5} \\
    \end{tabular}}
  \caption{Resumen de las dificultades más importantes en proyectos DGS}
  \label{tab:DificultadesDGS}
\end{table}

\subsection{Rol del jefe de proyecto}
\label{sec:ImportanciaJP}

La tarea de administración y gestión de un proyecto consiste en el uso de un conjunto de técnicas y herramientas con el objetivo de controlar los recursos de un proyecto y la correcta realización de las diferentes tareas de los miembros, para conseguir así una evolución constante y sin problemas en el desarrollo del servicio o producto software. Dicha tarea esta ligada a diferentes aspectos relevantes de un proyecto, como es la gestión del tiempo, el coste, la calidad y muchos otros. Por lo tanto, el rol de jefe de proyecto es de suma importancia, ya que controlará la consistencia y evolución del proyecto, además de asegurarse de que los objetivos y propósito del mismo se cumplan correctamente y sin incidencias. El jefe de proyecto deberá hacer frente a cualquier contratiempo que pueda surgir a lo largo del proceso de desarrollo del software \cite{colomo2014project}. Cabe destacar, que en la literatura se considera al proceso de gestionar un proyecto software una tarea esencial para alcanzar el éxito, sin embargo, también se considera dicho proceso como una tarea no sencilla, en la cual se necesitan grande conocimientos, habilidades y experiencia en el desarrollo de proyectos, según \cite{boehm1989theory} lo definió como la integración de la tecnología del software, la economía y las relaciones laborales en el contexto de un proyecto software. 

Por otro lado, si hablamos de la gestión de proyectos en un entorno global como es un proyecto DGS, dicha labor se complica aún más. Con la separación de los equipos de desarrolladores de un mismo proyecto en diferentes ubicaciones con sus diferencias temporales, culturales y lingüísticas, implica que el rol de jefe de proyectos DGS sea mucho más complicado, siendo necesaria la organización de trabajadores no conocidos personalmente a diferentes tareas, junto con la gestión de recursos ubicados en distintos lugares del mundo. Adicionalmente, dicha tarea se agrava con los desafíos en las llamadas 3 Ces, comunicación, coordinación y control, siendo un gran número los aspectos y desafíos que debe afrontar un jefe de proyectos DGS con el objetivo de alcanzar exitosamente la correcta colaboración entre los equipos de desarrolladores y la entrega del servicio o producto software al cliente. Debido a esta problemática, es necesaria una profunda educación a los estudiantes de ingeniería de software y a los futuros jefes de proyectos DGS tanto en este nuevo modelo de desarrollo de software como en el proceso más importante en este tipo de proyectos. Por lo que, será necesario un entrenamiento y enseñanza de aquellas habilidades, no solo técnicas, sino en especial aquellas no-técnicas (\emph{soft skills}) necesarias para afrontar con éxito los posibles desafíos que puedan surgir en un proyecto DGS.

\section{Habilidades necesarias en Desarrollo Global del Software}
\label{sec:HabilidadesDGS}

Como hemos dicho anteriormente, trabajar en un entorno distribuido como es un proyecto DGS, resulta complicado, ya que aparecen nuevas situaciones problemáticas debido a las dificultades en la comunicación y en la separación geográfica, temporal y socio-cultural. Además, en el proceso de gestión del proyecto esta situación se agrava siendo necesarios más conocimientos para cumplir correctamente con su trabajo. Por lo tanto, son necesarias ciertas habilidades para trabajar en dichas situaciones. Aunque, ambas, habilidades técnicas y no-técnicas son igual de importantes en un proyecto software, cabe destacar que las habilidades no-técnicas resultan ser más difíciles a la hora de enseñarlas y/o aprenderlas. Por lo tanto, en esta sección nos centraremos en listar y explicar cuales son las habilidades no-técnicas más importantes y necesarias cuando se trabaja en un entorno de desarrollo distribuido y cuando se lleva a cabo la labor de jefe de proyecto.

\subsection{Habilidades en el equipo de trabajo de Desarrollo Global del Software}
\label{sec:HabilidadesT}

Al igual que se citó anteriormente, el elevado nivel de fracaso en los proyectos DGS es debido a la falta de habilidades y competencias relacionadas con este nuevo modelo de desarrollo entre sus trabajadores. Es por esto, que son numerables los estudios que intentan recopilar el conjunto de habilidades necesarias para trabajar en un entorno DGS. Algunos de estos estudios son los siguientes:

En primer lugar, según \cite{bosnic2019assessing}, algunas de las habilidades más útiles en el desarrollo de software son \emph{hablar un idioma extranjero (en especial el inglés)}, \emph{capacidad de tener una cooperación local} y \emph{ser capaz de tomar decisiones}. En adicción a las anteriores, en un entorno distribuido se hacen necesarias nuevas habilidades como \emph{ser capaz de mantener una cooperación remota}, \emph{llevar a cabo una cooperación intercultural} y \emph{una cooperación con el cliente}.

Por otra parte, y de acuerdo con \cite{monasor2010training}, algunas de las habilidades que se deben promover en el entrenamiento de DGS son \emph{ser conscientes de todos los problemas posibles}, \emph{dominar protocolos para la correcta comunicación entre trabajadores (en especial con el uso de ordenadores)}, \emph{correcta comunicación oral y escrita a través de un idioma común}, \emph{conocer los códigos de ética de la organización} y \emph{llevar a cabo una correcta gestión del tiempo}.

En \cite{paasivaara2013teaching}, los autores indican que las habilidades más importantes que deben enseñarse para aprender a trabajar en DGS son \emph{comunicación regular entre los miembros de los equipos de desarrolladores distribuidos}, \emph{contribuir a la dinámica del equipo de trabajo}, \emph{saber como trabajar en equipos culturalmente divergentes}, \emph{gestión del tiempo} y \emph{saber trabajar con tecnologías de colaboración}.

Igualmente, un marco de trabajo para enseñar algunas habilidades de DGS se describe en \cite{damian2006instructional}; las principales habilidades consideradas son \emph{saber comunicarse mediante ordenadores}, \emph{desarrollo iterativo en las relaciones cliente-desarrollador a distancia} y \emph{gestión distribuida de proyectos}.

Por último, el modelo de competencias especificado en \cite{saldana2014skills} indica que las habilidades necesarias en un entorno DGS se pueden dividir en cuatro grupos, dependiendo del rol que se posee en el proyecto, estos roles pueden ser \emph{ingeniero de software}, líder de equipo, \emph{gestor de proyecto} y gerente de la unidad organizadora. Las habilidades que se señalan como necesarias en el entrenamiento de ingenieros de software DGS son \emph{comunicación síncrona y asíncrona}, \emph{identificación  y gestión de las necesidades del proyecto}, \emph{capacidad de solventar problemas técnicos}, \emph{conocimiento de técnicas avanzadas de comunicación distribuida}, \emph{capacidad de autoaprendizaje}, \emph{capacidad de mantener relaciones internacionales}, \emph{uso de tecnologías de la comunicación y la información}, \emph{capacidad de trabajar en un entorno global} y \emph{saber mantener una comunicación oral y escrita en inglés}.

Teniendo en cuenta todas las habilidades y competencias citadas anteriormente encontradas en la literatura, se muestra en la Tabla \ref{tab:SoftSkillsDGS} un resumen de aquellas soft skills más importantes para aprender a la hora de trabajar en un entorno DGS.

\begin{table}[htbp]
  \centering
  \resizebox{\textwidth}{!}{
    \begin{tabular}{lp{33.5em}c}
    	\rowcolor[rgb]{ .851,  .851,  .851} \multicolumn{1}{c}{\textbf{Soft Skills}} & \multicolumn{1}{c}{\textbf{Descripción de la soft skill}} & \textbf{Referencias} \\
        \rowcolor[rgb]{ .949,  .949,  .949} \multicolumn{1}{p{14.43em}}{\textit{Comunicación oral y escrita \newline{}en Inglés}} & \cellcolor[rgb]{ 1,  1,  1}Saber comunicarse oralmente y por escrito con los diferentes miembros de los equipos de trabajo en un idioma común, en especial el Inglés & \cellcolor[rgb]{ 1,  1,  1}\cite{bosnic2019assessing, monasor2010training, saldana2014skills} \\
        \rowcolor[rgb]{ .949,  .949,  .949} \multicolumn{1}{p{14.43em}}{\textit{Comunicación regular entre\newline{}los miembros de los equipos\newline{}de desarrolladores distribuidos}} & \cellcolor[rgb]{ 1,  1,  1}Mantener una comunicación fluida con los diferentes miembros de los equipos distribuidos, sabiendo en cada caso cuál es el mejor protocolo a utilizar (síncrono o asíncrono) y utilizando técnicas avanzadas & \cellcolor[rgb]{ 1,  1,  1}\cite{monasor2010training, paasivaara2013teaching, damian2006instructional, saldana2014skills} \\
        \rowcolor[rgb]{ .949,  .949,  .949} \multicolumn{1}{p{14.43em}}{\textit{Toma de decisiones y resolución\newline{}de problemas}} & \cellcolor[rgb]{ 1,  1,  1}Ser capaz de tomar las decisiones adecuadas para resolver los problemas técnicos que afectan directa o indirectamente a la evolución del proyecto, considerando todas las problemáticas posibles & \cellcolor[rgb]{ 1,  1,  1}\cite{bosnic2019assessing, monasor2010training, saldana2014skills} \\
        \rowcolor[rgb]{ .949,  .949,  .949} \textit{Gestión del tiempo} & \cellcolor[rgb]{ 1,  1,  1}Gestionar el tiempo empleado en el desarrollo de las tareas de los diferentes equipos de trabajo, en un esfuerzo por evitar problemas en el proyecto & \cellcolor[rgb]{ 1,  1,  1}\cite{monasor2010training, paasivaara2013teaching, damian2006instructional} \\
        \rowcolor[rgb]{ .949,  .949,  .949} \textit{Cooperación remota} & \cellcolor[rgb]{ 1,  1,  1}Colaborar tanto entre los miembros del mismo equipo como entre los diferentes equipos de trabajo para dinamizar el proyecto, utilizando también herramientas y tecnologías que ayuden en la cooperación & \cellcolor[rgb]{ 1,  1,  1}\cite{bosnic2019assessing, paasivaara2013teaching, damian2006instructional, saldana2014skills} \\
        \rowcolor[rgb]{ .949,  .949,  .949} \textit{Cooperación intercultural} & \cellcolor[rgb]{ 1,  1,  1}Saber cómo trabajar y cooperar con personas de diferentes culturas y tener la capacidad de relacionarse internacionalmente & \cellcolor[rgb]{ 1,  1,  1}\cite{bosnic2019assessing, paasivaara2013teaching, saldana2014skills} \\
        \rowcolor[rgb]{ .949,  .949,  .949} \textit{Cooperación con el cliente} & \cellcolor[rgb]{ 1,  1,  1}Cooperar con los clientes que nos han contratado para llevar a cabo el proyecto, identificando y gestionando los requisitos que quieren para tratar de completar el proyecto con éxito & \cellcolor[rgb]{ 1,  1,  1}\cite{bosnic2019assessing, damian2006instructional, saldana2014skills} \\
        \rowcolor[rgb]{ .949,  .949,  .949} \textit{Gestión del conocimiento} & \cellcolor[rgb]{ 1,  1,  1}Gestionar toda la información y el conocimiento que se crea en un proyecto distribuido y tratar de que llegue a todos los equipos distribuidos involucrados & \cellcolor[rgb]{ 1,  1,  1}\cite{saldana2014skills} \\
        \rowcolor[rgb]{ .949,  .949,  .949} \textit{Códigos de ética} & \cellcolor[rgb]{ 1,  1,  1}Conocer los código de ética de un proyecto distribuido y trabajar en \newline{}cumplimiento de estos & \cellcolor[rgb]{ 1,  1,  1}\cite{monasor2010training} \\
    \end{tabular}}
  \caption{Resumen de las soft skills más importantes para trabajar en proyectos DGS}
  \label{tab:SoftSkillsDGS}
\end{table}


\subsection{Habilidades en jefes de proyecto de Desarrollo Global del Software}
\label{sec:HabilidadesJP}

Una vez se ha llevado a cabo el anterior análisis sobre las habilidades y competencias necesarias para trabajar en un entorno DGS, a continuación nos centraremos en las competencias necesarias para llevar a cabo el rol de jefe de proyecto.

En primer lugar, los autores en \cite{verner2014risks}, indican que el riesgo es una de las partes más importantes en la gestión de un proyecto. Estos es debido a que en un proyecto de software distribuido los jefes de proyecto tienen que tener en cuenta áreas, en donde pueden surgir riesgos como puede ser \emph{la zona horaria}, \emph{la diferencia cultura}, \emph{la distancia geográfica} y \emph{la diferencia lingüística}, al igual que los procesos de \emph{coordinación}, \emph{comunicación} y \emph{control}.

En adicción a las anteriores consideraciones, la investigación en \cite{sutling2015understanding} ofrece un conjunto de habilidades las cuales deben de ser adquiridas por el futuro jefe de proyecto, para conseguir así lograr los objetivos de los proyectos que gestione. Las habilidades que se proponen son \emph{habilidades de comunicación}, \emph{habilidades para la construcción de entornos de equipo} y \emph{habilidades para la resolución de problemas}.

Por último, las competencias que se proponen en \cite{saldana2014skills} como necesarias en el entrenamiento de jefes de proyectos DGS son \emph{la toma de decisiones}, \emph{la gestión de reuniones}, \emph{el establecimiento de reglas para trabajar en un ambiente con datos compartidos}, \emph{saber recopilar, analizar e interpretar la información}, \emph{poseer una actitud positiva y una capacidad para motivar a los demás}, \emph{capacidad para organizar y planificar}, \emph{poseer iniciativa y liderazgo}, \emph{capacidad de resolver conflictos interpersonales}, \emph{identificación de competencias en Currículum-Vitae (CV)}, \emph{estimación de las necesidades y establecimiento de prioridades}.

Después de analizar y estudiar el conjunto de soft skills requeridas para llevar a cabo actividades de gestión de proyectos en entornos distribuidos, a continuación se muestra en la tabla \ref{tab:SoftSkillsJP} una agrupación de aquellas soft skills más importantes para aprender a gestionar un proyecto DGS correctamente.

\begin{table}[htbp]
  \centering
  \resizebox{\textwidth}{!}{
    \begin{tabular}{p{14.43em}p{33.93em}c}
    \rowcolor[rgb]{ .851,  .851,  .851} \multicolumn{1}{c}{\textbf{Soft Skills}} & \multicolumn{1}{c}{\textbf{Descripción de la soft skill}} & \textbf{Referencias} \\
    \rowcolor[rgb]{ .949,  .949,  .949} \textit{Actitud positiva y capacidad para motivar a otros} & \cellcolor[rgb]{ 1,  1,  1}Poseer la capacidad de animar a los diferentes miembros que componen el proyecto de software distribuido, aumentando así su capacidad de trabajo & \cellcolor[rgb]{ 1,  1,  1}\cite{saldana2014skills} \\
    \rowcolor[rgb]{ .949,  .949,  .949} \multicolumn{1}{l}{\textit{Iniciativa y liderazgo}} & \cellcolor[rgb]{ 1,  1,  1}Tener la capacidad de dirigir y gestionar el proyecto distribuido, tomando la iniciativa en las ideas & \cellcolor[rgb]{ 1,  1,  1}\cite{saldana2014skills} \\
    \rowcolor[rgb]{ .949,  .949,  .949} \textit{Toma de decisiones y resolución de problemas} & \cellcolor[rgb]{ 1,  1,  1}Poder tomar decisiones y así resolver los problemas (tanto técnicos como \newline{}interpersonales) que afectan directa o indirectamente a la evolución del \newline{}proyecto, considerando todas las posibles cuestiones involucradas & \cellcolor[rgb]{ 1,  1,  1}\cite{saldana2014skills, sutling2015understanding} \\
    \rowcolor[rgb]{ .949,  .949,  .949} \textit{Habilidades para la construcción de entornos de equipo} & \cellcolor[rgb]{ 1,  1,  1}Tener la capacidad de mejorar y fortalecer la relación y la comunicación con otros miembros y equipos de trabajo, creando así un ambiente de trabajo \newline{}confortable & \cellcolor[rgb]{ 1,  1,  1}\cite{sutling2015understanding} \\
    \rowcolor[rgb]{ .949,  .949,  .949} \multicolumn{1}{l}{\textit{Habilidades de comunicación}} & \cellcolor[rgb]{ 1,  1,  1}Poder comunicarse adecuadamente con los demás miembros del proyecto, así como con los clientes del mismo, eligiendo un idioma común en ambos \newline{}casos y teniendo en cuenta las diferencias culturales, geográficas y de zona \newline{}horaria & \cellcolor[rgb]{ 1,  1,  1}\cite{verner2014risks, sutling2015understanding} \\
    \rowcolor[rgb]{ .949,  .949,  .949} \multicolumn{1}{l}{\textit{Habilidades de coordinación}} & \cellcolor[rgb]{ 1,  1,  1}Saber combinar y gestionar todos los detalles técnicos y personales en las \newline{}diferentes tareas de las que se compone el proyecto distribuido, para \newline{}completarlo con éxito & \cellcolor[rgb]{ 1,  1,  1}\cite{saldana2014skills, verner2014risks} \\
    \rowcolor[rgb]{ .949,  .949,  .949} \multicolumn{1}{l}{\textit{Habilidades de control}} & \cellcolor[rgb]{ 1,  1,  1}Poder examinar la evolución del proyecto distribuido para comprobar si la \newline{}retroalimentación obtenida se corresponde con los requerimientos de los \newline{}clientes, recogiendo, analizando e interpretando la información obtenida & \cellcolor[rgb]{ 1,  1,  1}\cite{saldana2014skills, verner2014risks} \\
    \rowcolor[rgb]{ .949,  .949,  .949} \textit{Estimación de las necesidades y\newline{}establecimiento de prioridades} & \cellcolor[rgb]{ 1,  1,  1}Saber priorizar y estimar las diferentes tareas del proyecto distribuido en función de las demandas y requerimientos de los clientes en cada momento & \cellcolor[rgb]{ 1,  1,  1}\cite{saldana2014skills} \\
    \rowcolor[rgb]{ .949,  .949,  .949} \multicolumn{1}{l}{\textit{Gestión de reuniones}} & \cellcolor[rgb]{ 1,  1,  1}Tener la capacidad de llevar a cabo la gestión de la reunión, tanto con los \newline{}diferentes miembros del proyecto distribuido como con los clientes & \cellcolor[rgb]{ 1,  1,  1}\cite{saldana2014skills} \\
    \rowcolor[rgb]{ .949,  .949,  .949} \textit{Identificación de competencias \newline{}en CV} & \cellcolor[rgb]{ 1,  1,  1}Ser capaz de identificar cuáles son las habilidades necesarias para llevar a cabo una tarea, y decidir qué trabajador es el más apropiado para esa tarea, \newline{}según sus habilidades & \cellcolor[rgb]{ 1,  1,  1}\cite{saldana2014skills} \\
    \rowcolor[rgb]{ .949,  .949,  .949} \textit{Establecimiento de reglas para \newline{}trabajar en un ambiente con \newline{}datos compartidos} & \cellcolor[rgb]{ 1,  1,  1}Saber imponer las reglas adecuadas en la gestión de los datos compartidos \newline{}para que lleguen a todos los equipos de trabajo & \cellcolor[rgb]{ 1,  1,  1}\cite{saldana2014skills} \\
    \end{tabular}}
  \caption{Resumen de las soft skills más importantes para trabajar en proyectos DGS como jefe de proyecto}
  \label{tab:SoftSkillsJP}
\end{table}

\section{Gamificación e Inteligencia artificial}
\label{sec:Gamificacion}

\subsection{Juegos Serios}
\label{sec:JuegosSerios}

\subsection{Personalización de juegos mediante inteligencia artificial}
\label{sec:PersonalizacionJuegos}

\section{Trabajos relacionados con el tema}
\label{sec:TrabajosRelacionados}

\subsection{Juegos Serios para Desarrollo Global del Software}
\label{sec:JuegosSeriosDGS}

\subsection{Juegos Serios para Jefes de Proyecto}
\label{sec:JuegosSeriosJP}

\subsection{Juegos Serios para Jefes de Proyecto en Desarrollo Global del Software}
\label{sec:JuegosSeriosJPDGS}