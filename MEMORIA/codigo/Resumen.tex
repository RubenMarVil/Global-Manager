% ---------------------------------
%
% RESUMEN
%
% ---------------------------------

\pagestyle{plain}	% Página con numeración inferior al pie


% ----- Resumen del documento -----
\selectlanguage{spanish}	% Idioma del resumen(español)
\cleardoublepage			% Se incluye para modificar el contador de página antes de añadir bookmark
\pdfbookmark[0]{Resumen}{idx_resumen}	% Bookmark en PDF
\addcontentsline{toc}{chapter}{Resumen} % Añade al TOC

\begin{abstract}

En la actualidad, cada vez más empresas están introduciendo un nuevo modelo de desarrollo, el cual resulta ser más des-localizado que el modelo convencional, donde los miembros del proyecto pueden estar en distintos países. Esta tendencia, llamada Desarrollo Global de Software (DGS), está creciendo rápidamente debido a la globalización, sin embargo, conlleva que aparezcan nuevos riesgos en su gestión, los cuales pueden ser agrupados en tres bloques: comunicación, coordinación y control. Es por ello, que se necesitan a jefes de proyecto preparados para afrontar y solventar los problemas que puedan ocurrir, lo que requiere de ciertas habilidades técnicas y no técnicas (soft skills) para gestionar con éxito este tipo de proyectos.

Últimamente, ha cobrado gran importancia el uso de juegos serios para la enseñanza y el entrenamiento de un conjunto de conocimientos y habilidades especificas. Por lo tanto, en este proyecto se llevará a cabo el desarrollo de un juego serio para el entrenamiento de jefes de proyecto en habilidades no técnicas necesarias para afrontar con éxito la gestión de un proyecto de software global. Este juego contará con un módulo de Inteligencia Artificial con el fin de monitorizar las acciones del jugador y ajustar dinámicamente el desarrollo del mismo.

\end{abstract}


% ----- Abstract del documento -----
\selectlanguage{english}	% Idioma del abstract(inglés)
\cleardoublepage			% Se incluye para modificar el contador de página antes de añadir bookmark
\pdfbookmark[0]{Resumen}{idx_resumen}	% Bookmark en PDF
\addcontentsline{toc}{chapter}{Resumen} % Añade al TOC

\begin{abstract}

% TODO: Rellenar con el abstract

\end{abstract}


% ----- Ajuste del idioma para el resto del documento -----
\ifspanish
	\selectlanguage{spanish}	% Emplea idioma español
\else
	\selectlanguage{english}	% Emplea idioma inglés
\fi